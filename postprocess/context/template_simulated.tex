\documentclass{article}

\usepackage{postprocess/context/arxiv}

\usepackage[utf8]{inputenc} % allow utf-8 input
\usepackage[T1]{fontenc}    % use 8-bit T1 fonts
\usepackage{hyperref}       % hyperlinks
\usepackage{url}            % simple URL typesetting
\usepackage{booktabs}       % professional-quality tables
\usepackage{amsfonts}       % blackboard math symbols
\usepackage{nicefrac}       % compact symbols for 1/2, etc.
\usepackage{microtype}      % microtypography
\usepackage{lipsum}		% Can be removed after putting your text content
\usepackage{graphicx}
\usepackage{natbib}
\usepackage{doi}
\usepackage{float}
\usepackage{subcaption}



\title{[TITLE]}


\author{ \href{https://orcid.org/0000-0000-0000-0000}{\includegraphics[scale=0.06]{postprocess/context/orcid.pdf}\hspace{1mm}Causal Copilot}}
	

\renewcommand{\headeright}{Technical Report}
\renewcommand{\undertitle}{Technical Report}

\hypersetup{
pdftitle={[TITLE]},
pdfauthor={Causal Copilot},
pdfkeywords={Causal Discovery, Large Language Model, [ALGO]},
}

\begin{document}
\maketitle

\begin{abstract}
[ABSTRACT]
\end{abstract}

\keywords{Causal Discovery, Large Language Model, [ALGO]}

\raggedbottom
\section{Introduction}
[INTRO_INFO]

\section{Dataset Descriptions and EDA}
The following is a preview of our original dataset.

\begin{table}[H]
    \centering
    \caption{Dataset Preview}
    [DATA_PREVIEW]
\end{table}

\subsection{Data Properties}
We employ several statistical methods to identify data properties.

The shape of the data, data types, and missing values are assessed directly from the dataframe.
Linearity is evaluated using Ramsey’s RESET test, followed by the Benjamini \& Yekutieli procedure for multiple test correction.
Gaussian noise is assessed through the Shapiro-Wilk test, also applying the Benjamini \& Yekutieli procedure for multiple test correction.
Time-Series and Heterogeneity are derived from user queries.

Properties of the dataset we analyzed are listed below.

\begin{table}[H]
    \centering
    \caption{Data Properties}
[DATA_PROP_TABLE]
\end{table}


\subsection{Distribution Analysis}
The following figure shows distributions of different variables. The orange dash line represents the mean, 
and the black line represents the median. Variables are categorized into three types according to their distribution characteristics.

\begin{figure}[H]
\centering
\includegraphics[width=\linewidth]{[DIST_GRAPH]}
\caption{\label{fig:dist}Distribution Plots of Variables}
\end{figure}

[DIST_INFO]

\subsection{Correlation Analysis}

\begin{minipage}[t]{0.5\linewidth}
    [CORR_INFO]
\vfill
\end{minipage}
\hfill
\begin{minipage}[t]{0.5\linewidth}
    \begin{figure}[H]
        \centering
        \vspace{-1.5cm}
        \includegraphics[width=\linewidth]{[CORR_GRAPH]}
        \caption{\label{fig:corr}Correlation Heatmap of Variables}
    \end{figure}
\end{minipage}

\section{Discovery Procedure}
[DISCOVER_PROCESS]

\section{Results Summary}

\begin{figure}[H]
    \centering
    \begin{subfigure}{0.45\textwidth}
        \centering
        \vspace{-0.5cm}
        \includegraphics[width=\linewidth]{[RESULT_GRAPH0]}
        \vfill
        \caption{True Graph}
        \label{fig:sub1}
    \end{subfigure}
    \hspace{0.04\textwidth}
    \begin{subfigure}{0.45\textwidth}
        \centering
        \vspace{-0.5cm}
        \includegraphics[width=\linewidth]{[RESULT_GRAPH1]}
        \vfill
        \caption{Initial Graph}
        \label{fig:sub2}
    \end{subfigure}
    \caption{Graphs Comparision of [ALGO]}
    \label{fig:main}
\end{figure}

The above are the true graph and the result graph produced by our algorithm.

[RESULT_ANALYSIS]

\subsection{Graph Reliability Analysis}

\begin{figure}[H]
        \centering
        \vspace{-0.5cm}
        \includegraphics[width=0.8\linewidth]{[RESULT_GRAPH4]}
        \caption{Reliability Graph}
        \label{fig:sub3}
\end{figure}

Based on the confidence probability heatmap and background knowledge, we can analyze the reliability of our graph.

[RELIABILITY_ANALYSIS]

%\section{Metrics Evaluation}


\end{document}
